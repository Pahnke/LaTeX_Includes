% % % % \input{includes.tex}
% Used to import includes

\usepackage{enumitem}% http://ctan.org/pkg/enumitem
\newcommand{\nItem}[0]{\newpage \item}

%%%%%%%%%%%%%%%%%%%%%%%%%%%%%%%%%%%%%%%%%%%%%%%%%%%%%%%%%%%

\usepackage{titlesec}

\titleformat{\section}[wrap]
{\normalfont\bfseries}
{\thesection.}{0.5em}{}
\titlespacing{\section}{12pc}{1.5ex plus .1ex minus .2ex}{1pc}

\newcommand{\sectionbreak}[0]{\clearpage} % Section starts on a new page
\renewcommand{\thesubsection}[0]{\thesection.\alph{subsection}} % Subsections are letters

% Removes section numbers
%\makeatletter
%\renewcommand{\@seccntformat}[1]{}
%\makeatother

%%%%%%%%%%%%%%%%%%%%%%%%%%%%%%%%%%%%%%%%%%%%%%%%%%%%%%%%%%%

\usepackage{amsfonts}
\usepackage{amsmath}
\usepackage{amssymb}
\usepackage{mathtools}
\usepackage{cancel}
\usepackage{tikz} % Graphs

\usepackage{hyperref}
\usepackage{rotating}
\usepackage{fullpage}
\usepackage{fixltx2e} % Normal text sub and super scripts
\usepackage{minted} % Copy and paste code in
% \begin{minted}[mathescape, linenos]{python}
% \end{minted}

%%%%%%%%%%%%%%%%%%%%%%%%%%%%%%%%%%%%%%%%%%%%%%%%%%%%%%%%%%%

\usepackage[english]{babel}
\usepackage[utf8]{inputenc}
\usepackage{fancyhdr}

\fancypagestyle{firstPageStyle}
{
   \fancyhf{}
   \nameHeader
   \cfoot{\thepage} % Page No. center

} % Used to set style of first page

% \thispagestyle{firstPageStyle}
% Place on first page

%%%%%%%%%%%%%%%%%%%%%%%%%%%%%%%%%%%%%%%%%%%%%%%%%%%%%%%%%%%

\newcommand{\cid}[0]{01493016}
\newcommand{\lcid}[0]{CID: \cid}
\newcommand{\name}[0]{Ben David Pahnke}
\newcommand{\lname}[0]{Name: \name}
\newcommand{\degreeCode}[0]{GG41}
\newcommand{\ldegreeCode}[0]{Degree Code: \degreeCode}
\newcommand{\lcourseCode}[1]{Course Code: #1}

\newcommand{\nameHeader}[0]{\pagestyle{fancy}
\rhead{\name\\ \cid} \lhead{\leftmark}}
\newcommand{\courseHeader}[1]{\nameHeader
\lhead{\lcourseCode{#1}}}
\newcommand{\degreeHeader}[1]{\nameHeader
\lhead{\lcourseCode{#1} \\
\ldegreeCode}}

%%%%%%%%%%%%%%%%%%%%%%%%%%%%%%%%%%%%%%%%%%%%%%%%%%%%%%%%%%%

\newcommand{\tLine}[0]{  \\ \hline } % New Table line

%%%%%%%%%%%%%%%%%%%%%%%%%%%%%%%%%%%%%%%%%%%%%%%%%%%%%%%%%%%

\newcommand{\topCorn}[1]{\ulcorner #1 \urcorner}
\newcommand{\sPPair}[2]{\langle \langle #1, #2 \rangle \rangle}
\newcommand{\sPair}[2]{\langle #1, #2 \rangle}
\newcommand{\defEq}[0]{\stackrel{\text{def}}{=}} % = with def above it

%%%%%%%%%%%%%%%%%%%%%%%%%%%%%%%%%%%%%%%%%%%%%%%%%%%%%%%%%%%

\newcommand{\intAB}[2]{\int\limits^{#1}_{#2}}
\newcommand{\sumAB}[2]{\sum\limits_{#1}^{#2}}
\newcommand{\prodAB}[2]{\prod\limits_{#1}^{#2}}
\newcommand{\cupAB}[2]{\bigcup\limits_{#1}^{#2}}

%%%%%%%%%%%%%%%%%%%%%%%%%%%%%%%%%%%%%%%%%%%%%%%%%%%%%%%%%%%

\newcommand{\spTxt}[1]{\ \text{#1} \ }
\newcommand{\spImp}[0]{\ \implies \ }

%%%%%%%%%%%%%%%%%%%%%%%%%%%%%%%%%%%%%%%%%%%%%%%%%%%%%%%%%%%

\newcommand{\half}[0]{\frac{1}{2}}

%%%%%%%%%%%%%%%%%%%%%%%%%%%%%%%%%%%%%%%%%%%%%%%%%%%%%%%%%%%

\newcommand{\rang}[2]{ [#1  ... \ #2 ]\,} %Inclusive Range

%%%%%%%%%%%%%%%%%%%%%%%%%%%%%%%%%%%%%%%%%%%%%%%%%%%%%%%%%%%

\newcommand{\tL}{\text{L}}
\newcommand{\tP}{\text{P}}
\newcommand{\tE}{\text{E}}
\newcommand{\tQ}{\text{Q}}
\newcommand{\tR}{\text{R}}

\newcommand{\hpi}{\hat{\pi}}

%%%%%%%%%%%%%%%%%%%%%%%%%%%%%%%%%%%%%%%%%%%%%%%%%%%%%%%%%%%

\newcommand{\mR}[0]{\mathbb{R}} % real numbers
\newcommand{\mZ}[0]{\mathbb{Z}} % integers
\newcommand{\mN}[0]{\mathbb{N}} % natural numbers
\newcommand{\mQ}[0]{\mathbb{Q}} % rational numbers
\newcommand{\mC}[0]{\mathbb{C}} % complex numbers

%%%%%%%%%%%%%%%%%%%%%%%%%%%%%%%%%%%%%%%%%%%%%%%%%%%%%%%%%%%

\newcommand{\betaFunc}[2]{ \frac{ \Gamma (#1 + #2)}{\Gamma(#1)\Gamma(#2)}} % Beta Function

% Used to import includes

\usepackage{enumitem}% http://ctan.org/pkg/enumitem
\newcommand{\nItem}[0]{\newpage \item}

%%%%%%%%%%%%%%%%%%%%%%%%%%%%%%%%%%%%%%%%%%%%%%%%%%%%%%%%%%%

\usepackage{titlesec}

\titleformat{\section}[wrap]
{\normalfont\bfseries}
{\thesection.}{0.5em}{}
\titlespacing{\section}{12pc}{1.5ex plus .1ex minus .2ex}{1pc}

\newcommand{\sectionbreak}[0]{\clearpage} % Section starts on a new page
\renewcommand{\thesubsection}[0]{\thesection.\alph{subsection}} % Subsections are letters

% Removes section numbers
%\makeatletter
%\renewcommand{\@seccntformat}[1]{}
%\makeatother

%%%%%%%%%%%%%%%%%%%%%%%%%%%%%%%%%%%%%%%%%%%%%%%%%%%%%%%%%%%

\usepackage{amsfonts}
\usepackage{amsmath}
\usepackage{amssymb}
\usepackage{mathtools}
\usepackage{cancel}
\usepackage{tikz} % Graphs

\usepackage{hyperref}
\usepackage{rotating}
\usepackage{fullpage}
\usepackage{fixltx2e} % Normal text sub and super scripts
\usepackage{minted} % Copy and paste code in
% \begin{minted}[mathescape, linenos]{python}
% \end{minted}

%%%%%%%%%%%%%%%%%%%%%%%%%%%%%%%%%%%%%%%%%%%%%%%%%%%%%%%%%%%

\usepackage[english]{babel}
\usepackage[utf8]{inputenc}
\usepackage{fancyhdr}

\fancypagestyle{firstPageStyle}
{
   \fancyhf{}
   \nameHeader
   \cfoot{\thepage} % Page No. center

} % Used to set style of first page

% \thispagestyle{firstPageStyle}
% Place on first page

%%%%%%%%%%%%%%%%%%%%%%%%%%%%%%%%%%%%%%%%%%%%%%%%%%%%%%%%%%%

\newcommand{\cid}[0]{01493016}
\newcommand{\lcid}[0]{CID: \cid}
\newcommand{\name}[0]{Ben David Pahnke}
\newcommand{\lname}[0]{Name: \name}
\newcommand{\degreeCode}[0]{GG41}
\newcommand{\ldegreeCode}[0]{Degree Code: \degreeCode}
\newcommand{\lcourseCode}[1]{Course Code: #1}

\newcommand{\nameHeader}[0]{\pagestyle{fancy}
\rhead{\name\\ \cid} \lhead{\leftmark}}
\newcommand{\courseHeader}[1]{\nameHeader
\lhead{\lcourseCode{#1}}}
\newcommand{\degreeHeader}[1]{\nameHeader
\lhead{\lcourseCode{#1} \\
\ldegreeCode}}

%%%%%%%%%%%%%%%%%%%%%%%%%%%%%%%%%%%%%%%%%%%%%%%%%%%%%%%%%%%

\newcommand{\tLine}[0]{  \\ \hline } % New Table line

%%%%%%%%%%%%%%%%%%%%%%%%%%%%%%%%%%%%%%%%%%%%%%%%%%%%%%%%%%%

\newcommand{\topCorn}[1]{\ulcorner #1 \urcorner}
\newcommand{\sPPair}[2]{\langle \langle #1, #2 \rangle \rangle}
\newcommand{\sPair}[2]{\langle #1, #2 \rangle}
\newcommand{\defEq}[0]{\stackrel{\text{def}}{=}} % = with def above it

%%%%%%%%%%%%%%%%%%%%%%%%%%%%%%%%%%%%%%%%%%%%%%%%%%%%%%%%%%%

\newcommand{\intAB}[2]{\int\limits^{#1}_{#2}}
\newcommand{\sumAB}[2]{\sum\limits_{#1}^{#2}}
\newcommand{\prodAB}[2]{\prod\limits_{#1}^{#2}}
\newcommand{\cupAB}[2]{\bigcup\limits_{#1}^{#2}}

%%%%%%%%%%%%%%%%%%%%%%%%%%%%%%%%%%%%%%%%%%%%%%%%%%%%%%%%%%%

\newcommand{\spTxt}[1]{\ \text{#1} \ }
\newcommand{\spImp}[0]{\ \implies \ }

%%%%%%%%%%%%%%%%%%%%%%%%%%%%%%%%%%%%%%%%%%%%%%%%%%%%%%%%%%%

\newcommand{\half}[0]{\frac{1}{2}}

%%%%%%%%%%%%%%%%%%%%%%%%%%%%%%%%%%%%%%%%%%%%%%%%%%%%%%%%%%%

\newcommand{\rang}[2]{ [#1  ... \ #2 ]\,} %Inclusive Range

%%%%%%%%%%%%%%%%%%%%%%%%%%%%%%%%%%%%%%%%%%%%%%%%%%%%%%%%%%%

\newcommand{\tL}{\text{L}}
\newcommand{\tP}{\text{P}}
\newcommand{\tE}{\text{E}}
\newcommand{\tQ}{\text{Q}}
\newcommand{\tR}{\text{R}}

\newcommand{\hpi}{\hat{\pi}}

%%%%%%%%%%%%%%%%%%%%%%%%%%%%%%%%%%%%%%%%%%%%%%%%%%%%%%%%%%%

\newcommand{\mR}[0]{\mathbb{R}} % real numbers
\newcommand{\mZ}[0]{\mathbb{Z}} % integers
\newcommand{\mN}[0]{\mathbb{N}} % natural numbers
\newcommand{\mQ}[0]{\mathbb{Q}} % rational numbers
\newcommand{\mC}[0]{\mathbb{C}} % complex numbers

%%%%%%%%%%%%%%%%%%%%%%%%%%%%%%%%%%%%%%%%%%%%%%%%%%%%%%%%%%%

\newcommand{\betaFunc}[2]{ \frac{ \Gamma (#1 + #2)}{\Gamma(#1)\Gamma(#2)}} % Beta Function

% Used to import includes

\usepackage{enumitem}% http://ctan.org/pkg/enumitem
\newcommand{\nItem}[0]{\newpage \item}

%%%%%%%%%%%%%%%%%%%%%%%%%%%%%%%%%%%%%%%%%%%%%%%%%%%%%%%%%%%

\usepackage{titlesec}

\titleformat{\section}[wrap]
{\normalfont\bfseries}
{\thesection.}{0.5em}{}
\titlespacing{\section}{12pc}{1.5ex plus .1ex minus .2ex}{1pc}

\newcommand{\sectionbreak}[0]{\clearpage} % Section starts on a new page
\renewcommand{\thesubsection}[0]{\thesection.\alph{subsection}} % Subsections are letters

% Removes section numbers
%\makeatletter
%\renewcommand{\@seccntformat}[1]{}
%\makeatother

%%%%%%%%%%%%%%%%%%%%%%%%%%%%%%%%%%%%%%%%%%%%%%%%%%%%%%%%%%%

\usepackage{amsfonts}
\usepackage{amsmath}
\usepackage{amssymb}
\usepackage{mathtools}
\usepackage{cancel}
\usepackage{tikz} % Graphs

\usepackage{hyperref}
\usepackage{rotating}
\usepackage{fullpage}
\usepackage{fixltx2e} % Normal text sub and super scripts
\usepackage{minted} % Copy and paste code in
% \begin{minted}[mathescape, linenos]{python}
% \end{minted}

%%%%%%%%%%%%%%%%%%%%%%%%%%%%%%%%%%%%%%%%%%%%%%%%%%%%%%%%%%%

\usepackage[english]{babel}
\usepackage[utf8]{inputenc}
\usepackage{fancyhdr}

\fancypagestyle{firstPageStyle}
{
   \fancyhf{}
   \nameHeader
   \cfoot{\thepage} % Page No. center

} % Used to set style of first page

% \thispagestyle{firstPageStyle}
% Place on first page

%%%%%%%%%%%%%%%%%%%%%%%%%%%%%%%%%%%%%%%%%%%%%%%%%%%%%%%%%%%

\newcommand{\cid}[0]{01493016}
\newcommand{\lcid}[0]{CID: \cid}
\newcommand{\name}[0]{Ben David Pahnke}
\newcommand{\lname}[0]{Name: \name}
\newcommand{\degreeCode}[0]{GG41}
\newcommand{\ldegreeCode}[0]{Degree Code: \degreeCode}
\newcommand{\lcourseCode}[1]{Course Code: #1}

\newcommand{\nameHeader}[0]{\pagestyle{fancy}
\rhead{\name\\ \cid} \lhead{\leftmark}}
\newcommand{\courseHeader}[1]{\nameHeader
\lhead{\lcourseCode{#1}}}
\newcommand{\degreeHeader}[1]{\nameHeader
\lhead{\lcourseCode{#1} \\
\ldegreeCode}}

%%%%%%%%%%%%%%%%%%%%%%%%%%%%%%%%%%%%%%%%%%%%%%%%%%%%%%%%%%%

\newcommand{\tLine}[0]{  \\ \hline } % New Table line

%%%%%%%%%%%%%%%%%%%%%%%%%%%%%%%%%%%%%%%%%%%%%%%%%%%%%%%%%%%

\newcommand{\topCorn}[1]{\ulcorner #1 \urcorner}
\newcommand{\sPPair}[2]{\langle \langle #1, #2 \rangle \rangle}
\newcommand{\sPair}[2]{\langle #1, #2 \rangle}
\newcommand{\defEq}[0]{\stackrel{\text{def}}{=}} % = with def above it

%%%%%%%%%%%%%%%%%%%%%%%%%%%%%%%%%%%%%%%%%%%%%%%%%%%%%%%%%%%

\newcommand{\intAB}[2]{\int\limits^{#1}_{#2}}
\newcommand{\sumAB}[2]{\sum\limits_{#1}^{#2}}
\newcommand{\prodAB}[2]{\prod\limits_{#1}^{#2}}
\newcommand{\cupAB}[2]{\bigcup\limits_{#1}^{#2}}

%%%%%%%%%%%%%%%%%%%%%%%%%%%%%%%%%%%%%%%%%%%%%%%%%%%%%%%%%%%

\newcommand{\spTxt}[1]{\ \text{#1} \ }
\newcommand{\spImp}[0]{\ \implies \ }

%%%%%%%%%%%%%%%%%%%%%%%%%%%%%%%%%%%%%%%%%%%%%%%%%%%%%%%%%%%

\newcommand{\half}[0]{\frac{1}{2}}

%%%%%%%%%%%%%%%%%%%%%%%%%%%%%%%%%%%%%%%%%%%%%%%%%%%%%%%%%%%

\newcommand{\rang}[2]{ [#1  ... \ #2 ]\,} %Inclusive Range

%%%%%%%%%%%%%%%%%%%%%%%%%%%%%%%%%%%%%%%%%%%%%%%%%%%%%%%%%%%

\newcommand{\tL}{\text{L}}
\newcommand{\tP}{\text{P}}
\newcommand{\tE}{\text{E}}
\newcommand{\tQ}{\text{Q}}
\newcommand{\tR}{\text{R}}

\newcommand{\hpi}{\hat{\pi}}

%%%%%%%%%%%%%%%%%%%%%%%%%%%%%%%%%%%%%%%%%%%%%%%%%%%%%%%%%%%

\newcommand{\mR}[0]{\mathbb{R}} % real numbers
\newcommand{\mZ}[0]{\mathbb{Z}} % integers
\newcommand{\mN}[0]{\mathbb{N}} % natural numbers
\newcommand{\mQ}[0]{\mathbb{Q}} % rational numbers
\newcommand{\mC}[0]{\mathbb{C}} % complex numbers

%%%%%%%%%%%%%%%%%%%%%%%%%%%%%%%%%%%%%%%%%%%%%%%%%%%%%%%%%%%

\newcommand{\betaFunc}[2]{ \frac{ \Gamma (#1 + #2)}{\Gamma(#1)\Gamma(#2)}} % Beta Function

% Used to import includes

%%%%%%%%%%%%%%%%%%%%%%%%%%%%%%%%%%%%%%%%%%%%%%%%%%%%%%%%%%%

\usepackage{enumitem}% http://ctan.org/pkg/enumitem
\newcommand{\nItem}[0]{\newpage \item}
\newcommand{\nitem}{\nItem}
\newcommand{\letEnum}{\begin{enumerate}[label = \alph*)]}
\newcommand{\numEnum}{\begin{enumerate}[label = \arabic*)]}
\newcommand{\eEnum}{\end{enumerate}}

%%%%%%%%%%%%%%%%%%%%%%%%%%%%%%%%%%%%%%%%%%%%%%%%%%%%%%%%%%%

\usepackage{titlesec}

\titleformat{\section}[wrap]
{\normalfont\bfseries}
{\thesection.}{0.5em}{}
\titlespacing{\section}{12pc}{1.5ex plus .1ex minus .2ex}{1pc}

\newcommand{\sectionbreak}[0]{\clearpage} % Section starts on a new page
\renewcommand{\thesubsection}[0]{\thesection.\alph{subsection}} % Subsections are letters

\allowdisplaybreaks

% Removes section numbers
%\makeatletter
%\renewcommand{\@seccntformat}[1]{}
%\makeatother

%%%%%%%%%%%%%%%%%%%%%%%%%%%%%%%%%%%%%%%%%%%%%%%%%%%%%%%%%%%

\usepackage{amsfonts}
\usepackage{amsmath}
\usepackage{amssymb}
\usepackage{mathtools}
\usepackage{cancel}
\usepackage{tikz} % Graphs

\usepackage{hyperref}
\usepackage{rotating}
\usepackage{fullpage}
\usepackage{fixltx2e} % Normal text sub and super scripts
\usepackage{minted} % Copy and paste code in
% \begin{minted}[mathescape, linenos]{python}
% \end{minted}
% \inputminted[mathescape, linenos]{python}{FILENAME.py}
\usepackage{graphicx}
%\includegraphics[width=\textwidth]{FILE NAME}
\usepackage{pdfpages}
%\includepdf[page={page number}]{filename}
%\includepdf[page=-]{filename}

%%%%%%%%%%%%%%%%%%%%%%%%%%%%%%%%%%%%%%%%%%%%%%%%%%%%%%%%%%%

\usepackage[english]{babel}
\usepackage[utf8]{inputenc}
\usepackage{fancyhdr}

\usepackage[
	headheight=120pt,% gap between header and text, gap between line and text
]{geometry}

\fancypagestyle{headerPageStyle}
{
   \fancyhf{}
   \cfoot{\thepage} % Page No. center
}

%%%%%%%%%%%%%%%%%%%%%%%%%%%%%%%%%%%%%%%%%%%%%%%%%%%%%%%%%%%

\newcommand{\cid}[0]{01493016}
\newcommand{\lcid}[0]{CID: \cid}
\newcommand{\name}[0]{Ben David Pahnke}
\newcommand{\lname}[0]{Name: \name}
\newcommand{\degreeCode}[0]{GG41}
\newcommand{\ldegreeCode}[0]{Degree Code: \degreeCode}
\newcommand{\lcourseCode}[1]{Course Code: #1}

\newcommand{\nameHeader}[0]{\pagestyle{headerPageStyle}
\rhead{\name\\ \cid} \lhead{\leftmark}}
\newcommand{\degreeHeader}[0]{\nameHeader
\lhead{\ldegreeCode}}
\newcommand{\courseHeader}[1]{\nameHeader
\lhead{\lcourseCode{#1} \\ \ldegreeCode}}

%%%%%%%%%%%%%%%%%%%%%%%%%%%%%%%%%%%%%%%%%%%%%%%%%%%%%%%%%%%

\newcommand{\tLine}[0]{  \\ \hline } % New Table line

%%%%%%%%%%%%%%%%%%%%%%%%%%%%%%%%%%%%%%%%%%%%%%%%%%%%%%%%%%%

\newcommand{\topCorn}[1]{\ulcorner #1 \urcorner}
\newcommand{\sPPair}[2]{\langle \langle #1, #2 \rangle \rangle}
\newcommand{\sPair}[2]{\langle #1, #2 \rangle}
\newcommand{\defEq}[0]{\stackrel{\text{def}}{=}} % = with def above it

%%%%%%%%%%%%%%%%%%%%%%%%%%%%%%%%%%%%%%%%%%%%%%%%%%%%%%%%%%%

\newcommand{\intAB}[2]{\int\limits^{#1}_{#2}}
\newcommand{\sumAB}[2]{\sum\limits_{#1}^{#2}}
\newcommand{\prodAB}[2]{\prod\limits_{#1}^{#2}}
\newcommand{\cupAB}[2]{\bigcup\limits_{#1}^{#2}}

%%%%%%%%%%%%%%%%%%%%%%%%%%%%%%%%%%%%%%%%%%%%%%%%%%%%%%%%%%%

\newcommand{\spTxt}[1]{\ \text{#1} \ }
\newcommand{\spImp}[0]{\ \implies \ }

%%%%%%%%%%%%%%%%%%%%%%%%%%%%%%%%%%%%%%%%%%%%%%%%%%%%%%%%%%%

\newcommand{\half}[0]{\frac{1}{2}}

%%%%%%%%%%%%%%%%%%%%%%%%%%%%%%%%%%%%%%%%%%%%%%%%%%%%%%%%%%%

\newcommand{\rang}[2]{ [#1  ... \ #2 ]\,} %Inclusive Range

%%%%%%%%%%%%%%%%%%%%%%%%%%%%%%%%%%%%%%%%%%%%%%%%%%%%%%%%%%%

\newcommand{\tL}{\text{L}}
\newcommand{\tP}{\text{P}}
\newcommand{\tE}{\text{E}}
\newcommand{\tQ}{\text{Q}}
\newcommand{\tR}{\text{R}}

\newcommand{\hpi}{\hat{\pi}}

%%%%%%%%%%%%%%%%%%%%%%%%%%%%%%%%%%%%%%%%%%%%%%%%%%%%%%%%%%%

\newcommand{\mR}[0]{\mathbb{R}} % real numbers
\newcommand{\mZ}[0]{\mathbb{Z}} % integers
\newcommand{\mN}[0]{\mathbb{N}} % natural numbers
\newcommand{\mQ}[0]{\mathbb{Q}} % rational numbers
\newcommand{\mC}[0]{\mathbb{C}} % complex numbers

%%%%%%%%%%%%%%%%%%%%%%%%%%%%%%%%%%%%%%%%%%%%%%%%%%%%%%%%%%%

\newcommand{\betaFunc}[2]{ \frac{ \Gamma (#1 + #2)}{\Gamma(#1)\Gamma(#2)}} % Beta Function
