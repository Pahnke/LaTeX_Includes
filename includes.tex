% \documentclass[12pt,twoside]{article}
% % % % \input{includes.tex}
% Used to import includes

\usepackage{enumitem}% http://ctan.org/pkg/enumitem
\newcommand{\nItem}[0]{\newpage \item}

%%%%%%%%%%%%%%%%%%%%%%%%%%%%%%%%%%%%%%%%%%%%%%%%%%%%%%%%%%%

\usepackage{titlesec}

\titleformat{\section}[wrap]
{\normalfont\bfseries}
{\thesection.}{0.5em}{}
\titlespacing{\section}{12pc}{1.5ex plus .1ex minus .2ex}{1pc}

\newcommand{\sectionbreak}[0]{\clearpage} % Section starts on a new page
\renewcommand{\thesubsection}[0]{\thesection.\alph{subsection}} % Subsections are letters

% Removes section numbers
%\makeatletter
%\renewcommand{\@seccntformat}[1]{}
%\makeatother

%%%%%%%%%%%%%%%%%%%%%%%%%%%%%%%%%%%%%%%%%%%%%%%%%%%%%%%%%%%

\usepackage{amsfonts}
\usepackage{amsmath}
\usepackage{amssymb}
\usepackage{mathtools}
\usepackage{cancel}
\usepackage{tikz} % Graphs

\usepackage{hyperref}
\usepackage{rotating}
\usepackage{fullpage}
\usepackage{fixltx2e} % Normal text sub and super scripts
\usepackage{minted} % Copy and paste code in
% \begin{minted}[mathescape, linenos]{python}
% \end{minted}

%%%%%%%%%%%%%%%%%%%%%%%%%%%%%%%%%%%%%%%%%%%%%%%%%%%%%%%%%%%

\usepackage[english]{babel}
\usepackage[utf8]{inputenc}
\usepackage{fancyhdr}

\fancypagestyle{firstPageStyle}
{
   \fancyhf{}
   \nameHeader
   \cfoot{\thepage} % Page No. center

} % Used to set style of first page

% \thispagestyle{firstPageStyle}
% Place on first page

%%%%%%%%%%%%%%%%%%%%%%%%%%%%%%%%%%%%%%%%%%%%%%%%%%%%%%%%%%%

\newcommand{\cid}[0]{01493016}
\newcommand{\lcid}[0]{CID: \cid}
\newcommand{\name}[0]{Ben David Pahnke}
\newcommand{\lname}[0]{Name: \name}
\newcommand{\degreeCode}[0]{GG41}
\newcommand{\ldegreeCode}[0]{Degree Code: \degreeCode}
\newcommand{\lcourseCode}[1]{Course Code: #1}

\newcommand{\nameHeader}[0]{\pagestyle{fancy}
\rhead{\name\\ \cid} \lhead{\leftmark}}
\newcommand{\courseHeader}[1]{\nameHeader
\lhead{\lcourseCode{#1}}}
\newcommand{\degreeHeader}[1]{\nameHeader
\lhead{\lcourseCode{#1} \\
\ldegreeCode}}

%%%%%%%%%%%%%%%%%%%%%%%%%%%%%%%%%%%%%%%%%%%%%%%%%%%%%%%%%%%

\newcommand{\tLine}[0]{  \\ \hline } % New Table line

%%%%%%%%%%%%%%%%%%%%%%%%%%%%%%%%%%%%%%%%%%%%%%%%%%%%%%%%%%%

\newcommand{\topCorn}[1]{\ulcorner #1 \urcorner}
\newcommand{\sPPair}[2]{\langle \langle #1, #2 \rangle \rangle}
\newcommand{\sPair}[2]{\langle #1, #2 \rangle}
\newcommand{\defEq}[0]{\stackrel{\text{def}}{=}} % = with def above it

%%%%%%%%%%%%%%%%%%%%%%%%%%%%%%%%%%%%%%%%%%%%%%%%%%%%%%%%%%%

\newcommand{\intAB}[2]{\int\limits^{#1}_{#2}}
\newcommand{\sumAB}[2]{\sum\limits_{#1}^{#2}}
\newcommand{\prodAB}[2]{\prod\limits_{#1}^{#2}}
\newcommand{\cupAB}[2]{\bigcup\limits_{#1}^{#2}}

%%%%%%%%%%%%%%%%%%%%%%%%%%%%%%%%%%%%%%%%%%%%%%%%%%%%%%%%%%%

\newcommand{\spTxt}[1]{\ \text{#1} \ }
\newcommand{\spImp}[0]{\ \implies \ }

%%%%%%%%%%%%%%%%%%%%%%%%%%%%%%%%%%%%%%%%%%%%%%%%%%%%%%%%%%%

\newcommand{\half}[0]{\frac{1}{2}}

%%%%%%%%%%%%%%%%%%%%%%%%%%%%%%%%%%%%%%%%%%%%%%%%%%%%%%%%%%%

\newcommand{\rang}[2]{ [#1  ... \ #2 ]\,} %Inclusive Range

%%%%%%%%%%%%%%%%%%%%%%%%%%%%%%%%%%%%%%%%%%%%%%%%%%%%%%%%%%%

\newcommand{\tL}{\text{L}}
\newcommand{\tP}{\text{P}}
\newcommand{\tE}{\text{E}}
\newcommand{\tQ}{\text{Q}}
\newcommand{\tR}{\text{R}}

\newcommand{\hpi}{\hat{\pi}}

%%%%%%%%%%%%%%%%%%%%%%%%%%%%%%%%%%%%%%%%%%%%%%%%%%%%%%%%%%%

\newcommand{\mR}[0]{\mathbb{R}} % real numbers
\newcommand{\mZ}[0]{\mathbb{Z}} % integers
\newcommand{\mN}[0]{\mathbb{N}} % natural numbers
\newcommand{\mQ}[0]{\mathbb{Q}} % rational numbers
\newcommand{\mC}[0]{\mathbb{C}} % complex numbers

%%%%%%%%%%%%%%%%%%%%%%%%%%%%%%%%%%%%%%%%%%%%%%%%%%%%%%%%%%%

\newcommand{\betaFunc}[2]{ \frac{ \Gamma (#1 + #2)}{\Gamma(#1)\Gamma(#2)}} % Beta Function

% Used to import includes

\usepackage{enumitem}% http://ctan.org/pkg/enumitem
\newcommand{\nItem}[0]{\newpage \item}

%%%%%%%%%%%%%%%%%%%%%%%%%%%%%%%%%%%%%%%%%%%%%%%%%%%%%%%%%%%

\usepackage{titlesec}

\titleformat{\section}[wrap]
{\normalfont\bfseries}
{\thesection.}{0.5em}{}
\titlespacing{\section}{12pc}{1.5ex plus .1ex minus .2ex}{1pc}

\newcommand{\sectionbreak}[0]{\clearpage} % Section starts on a new page
\renewcommand{\thesubsection}[0]{\thesection.\alph{subsection}} % Subsections are letters

% Removes section numbers
%\makeatletter
%\renewcommand{\@seccntformat}[1]{}
%\makeatother

%%%%%%%%%%%%%%%%%%%%%%%%%%%%%%%%%%%%%%%%%%%%%%%%%%%%%%%%%%%

\usepackage{amsfonts}
\usepackage{amsmath}
\usepackage{amssymb}
\usepackage{mathtools}
\usepackage{cancel}
\usepackage{tikz} % Graphs

\usepackage{hyperref}
\usepackage{rotating}
\usepackage{fullpage}
\usepackage{fixltx2e} % Normal text sub and super scripts
\usepackage{minted} % Copy and paste code in
% \begin{minted}[mathescape, linenos]{python}
% \end{minted}

%%%%%%%%%%%%%%%%%%%%%%%%%%%%%%%%%%%%%%%%%%%%%%%%%%%%%%%%%%%

\usepackage[english]{babel}
\usepackage[utf8]{inputenc}
\usepackage{fancyhdr}

\fancypagestyle{firstPageStyle}
{
   \fancyhf{}
   \nameHeader
   \cfoot{\thepage} % Page No. center

} % Used to set style of first page

% \thispagestyle{firstPageStyle}
% Place on first page

%%%%%%%%%%%%%%%%%%%%%%%%%%%%%%%%%%%%%%%%%%%%%%%%%%%%%%%%%%%

\newcommand{\cid}[0]{01493016}
\newcommand{\lcid}[0]{CID: \cid}
\newcommand{\name}[0]{Ben David Pahnke}
\newcommand{\lname}[0]{Name: \name}
\newcommand{\degreeCode}[0]{GG41}
\newcommand{\ldegreeCode}[0]{Degree Code: \degreeCode}
\newcommand{\lcourseCode}[1]{Course Code: #1}

\newcommand{\nameHeader}[0]{\pagestyle{fancy}
\rhead{\name\\ \cid} \lhead{\leftmark}}
\newcommand{\courseHeader}[1]{\nameHeader
\lhead{\lcourseCode{#1}}}
\newcommand{\degreeHeader}[1]{\nameHeader
\lhead{\lcourseCode{#1} \\
\ldegreeCode}}

%%%%%%%%%%%%%%%%%%%%%%%%%%%%%%%%%%%%%%%%%%%%%%%%%%%%%%%%%%%

\newcommand{\tLine}[0]{  \\ \hline } % New Table line

%%%%%%%%%%%%%%%%%%%%%%%%%%%%%%%%%%%%%%%%%%%%%%%%%%%%%%%%%%%

\newcommand{\topCorn}[1]{\ulcorner #1 \urcorner}
\newcommand{\sPPair}[2]{\langle \langle #1, #2 \rangle \rangle}
\newcommand{\sPair}[2]{\langle #1, #2 \rangle}
\newcommand{\defEq}[0]{\stackrel{\text{def}}{=}} % = with def above it

%%%%%%%%%%%%%%%%%%%%%%%%%%%%%%%%%%%%%%%%%%%%%%%%%%%%%%%%%%%

\newcommand{\intAB}[2]{\int\limits^{#1}_{#2}}
\newcommand{\sumAB}[2]{\sum\limits_{#1}^{#2}}
\newcommand{\prodAB}[2]{\prod\limits_{#1}^{#2}}
\newcommand{\cupAB}[2]{\bigcup\limits_{#1}^{#2}}

%%%%%%%%%%%%%%%%%%%%%%%%%%%%%%%%%%%%%%%%%%%%%%%%%%%%%%%%%%%

\newcommand{\spTxt}[1]{\ \text{#1} \ }
\newcommand{\spImp}[0]{\ \implies \ }

%%%%%%%%%%%%%%%%%%%%%%%%%%%%%%%%%%%%%%%%%%%%%%%%%%%%%%%%%%%

\newcommand{\half}[0]{\frac{1}{2}}

%%%%%%%%%%%%%%%%%%%%%%%%%%%%%%%%%%%%%%%%%%%%%%%%%%%%%%%%%%%

\newcommand{\rang}[2]{ [#1  ... \ #2 ]\,} %Inclusive Range

%%%%%%%%%%%%%%%%%%%%%%%%%%%%%%%%%%%%%%%%%%%%%%%%%%%%%%%%%%%

\newcommand{\tL}{\text{L}}
\newcommand{\tP}{\text{P}}
\newcommand{\tE}{\text{E}}
\newcommand{\tQ}{\text{Q}}
\newcommand{\tR}{\text{R}}

\newcommand{\hpi}{\hat{\pi}}

%%%%%%%%%%%%%%%%%%%%%%%%%%%%%%%%%%%%%%%%%%%%%%%%%%%%%%%%%%%

\newcommand{\mR}[0]{\mathbb{R}} % real numbers
\newcommand{\mZ}[0]{\mathbb{Z}} % integers
\newcommand{\mN}[0]{\mathbb{N}} % natural numbers
\newcommand{\mQ}[0]{\mathbb{Q}} % rational numbers
\newcommand{\mC}[0]{\mathbb{C}} % complex numbers

%%%%%%%%%%%%%%%%%%%%%%%%%%%%%%%%%%%%%%%%%%%%%%%%%%%%%%%%%%%

\newcommand{\betaFunc}[2]{ \frac{ \Gamma (#1 + #2)}{\Gamma(#1)\Gamma(#2)}} % Beta Function

% Used to import includes

\usepackage{enumitem}% http://ctan.org/pkg/enumitem
\newcommand{\nItem}[0]{\newpage \item}

%%%%%%%%%%%%%%%%%%%%%%%%%%%%%%%%%%%%%%%%%%%%%%%%%%%%%%%%%%%

\usepackage{titlesec}

\titleformat{\section}[wrap]
{\normalfont\bfseries}
{\thesection.}{0.5em}{}
\titlespacing{\section}{12pc}{1.5ex plus .1ex minus .2ex}{1pc}

\newcommand{\sectionbreak}[0]{\clearpage} % Section starts on a new page
\renewcommand{\thesubsection}[0]{\thesection.\alph{subsection}} % Subsections are letters

% Removes section numbers
%\makeatletter
%\renewcommand{\@seccntformat}[1]{}
%\makeatother

%%%%%%%%%%%%%%%%%%%%%%%%%%%%%%%%%%%%%%%%%%%%%%%%%%%%%%%%%%%

\usepackage{amsfonts}
\usepackage{amsmath}
\usepackage{amssymb}
\usepackage{mathtools}
\usepackage{cancel}
\usepackage{tikz} % Graphs

\usepackage{hyperref}
\usepackage{rotating}
\usepackage{fullpage}
\usepackage{fixltx2e} % Normal text sub and super scripts
\usepackage{minted} % Copy and paste code in
% \begin{minted}[mathescape, linenos]{python}
% \end{minted}

%%%%%%%%%%%%%%%%%%%%%%%%%%%%%%%%%%%%%%%%%%%%%%%%%%%%%%%%%%%

\usepackage[english]{babel}
\usepackage[utf8]{inputenc}
\usepackage{fancyhdr}

\fancypagestyle{firstPageStyle}
{
   \fancyhf{}
   \nameHeader
   \cfoot{\thepage} % Page No. center

} % Used to set style of first page

% \thispagestyle{firstPageStyle}
% Place on first page

%%%%%%%%%%%%%%%%%%%%%%%%%%%%%%%%%%%%%%%%%%%%%%%%%%%%%%%%%%%

\newcommand{\cid}[0]{01493016}
\newcommand{\lcid}[0]{CID: \cid}
\newcommand{\name}[0]{Ben David Pahnke}
\newcommand{\lname}[0]{Name: \name}
\newcommand{\degreeCode}[0]{GG41}
\newcommand{\ldegreeCode}[0]{Degree Code: \degreeCode}
\newcommand{\lcourseCode}[1]{Course Code: #1}

\newcommand{\nameHeader}[0]{\pagestyle{fancy}
\rhead{\name\\ \cid} \lhead{\leftmark}}
\newcommand{\courseHeader}[1]{\nameHeader
\lhead{\lcourseCode{#1}}}
\newcommand{\degreeHeader}[1]{\nameHeader
\lhead{\lcourseCode{#1} \\
\ldegreeCode}}

%%%%%%%%%%%%%%%%%%%%%%%%%%%%%%%%%%%%%%%%%%%%%%%%%%%%%%%%%%%

\newcommand{\tLine}[0]{  \\ \hline } % New Table line

%%%%%%%%%%%%%%%%%%%%%%%%%%%%%%%%%%%%%%%%%%%%%%%%%%%%%%%%%%%

\newcommand{\topCorn}[1]{\ulcorner #1 \urcorner}
\newcommand{\sPPair}[2]{\langle \langle #1, #2 \rangle \rangle}
\newcommand{\sPair}[2]{\langle #1, #2 \rangle}
\newcommand{\defEq}[0]{\stackrel{\text{def}}{=}} % = with def above it

%%%%%%%%%%%%%%%%%%%%%%%%%%%%%%%%%%%%%%%%%%%%%%%%%%%%%%%%%%%

\newcommand{\intAB}[2]{\int\limits^{#1}_{#2}}
\newcommand{\sumAB}[2]{\sum\limits_{#1}^{#2}}
\newcommand{\prodAB}[2]{\prod\limits_{#1}^{#2}}
\newcommand{\cupAB}[2]{\bigcup\limits_{#1}^{#2}}

%%%%%%%%%%%%%%%%%%%%%%%%%%%%%%%%%%%%%%%%%%%%%%%%%%%%%%%%%%%

\newcommand{\spTxt}[1]{\ \text{#1} \ }
\newcommand{\spImp}[0]{\ \implies \ }

%%%%%%%%%%%%%%%%%%%%%%%%%%%%%%%%%%%%%%%%%%%%%%%%%%%%%%%%%%%

\newcommand{\half}[0]{\frac{1}{2}}

%%%%%%%%%%%%%%%%%%%%%%%%%%%%%%%%%%%%%%%%%%%%%%%%%%%%%%%%%%%

\newcommand{\rang}[2]{ [#1  ... \ #2 ]\,} %Inclusive Range

%%%%%%%%%%%%%%%%%%%%%%%%%%%%%%%%%%%%%%%%%%%%%%%%%%%%%%%%%%%

\newcommand{\tL}{\text{L}}
\newcommand{\tP}{\text{P}}
\newcommand{\tE}{\text{E}}
\newcommand{\tQ}{\text{Q}}
\newcommand{\tR}{\text{R}}

\newcommand{\hpi}{\hat{\pi}}

%%%%%%%%%%%%%%%%%%%%%%%%%%%%%%%%%%%%%%%%%%%%%%%%%%%%%%%%%%%

\newcommand{\mR}[0]{\mathbb{R}} % real numbers
\newcommand{\mZ}[0]{\mathbb{Z}} % integers
\newcommand{\mN}[0]{\mathbb{N}} % natural numbers
\newcommand{\mQ}[0]{\mathbb{Q}} % rational numbers
\newcommand{\mC}[0]{\mathbb{C}} % complex numbers

%%%%%%%%%%%%%%%%%%%%%%%%%%%%%%%%%%%%%%%%%%%%%%%%%%%%%%%%%%%

\newcommand{\betaFunc}[2]{ \frac{ \Gamma (#1 + #2)}{\Gamma(#1)\Gamma(#2)}} % Beta Function

% Used to import includes

%%%%%%%%%%%%%%%%%%%%%%%%%%%%%%%%%%%%%%%%%%%%%%%%%%%%%%%%%%%

\usepackage{enumitem}% http://ctan.org/pkg/enumitem
\newcommand{\nItem}[0]{\newpage \item}
\newcommand{\nitem}[0]{\nItem}
\newcommand{\letEnum}[0]{\begin{enumerate}[label = \alph*)]}
\newcommand{\numEnum}[0]{\begin{enumerate}[label = \arabic*)]}
\newcommand{\eEnum}[0]{\end{enumerate}}

%%%%%%%%%%%%%%%%%%%%%%%%%%%%%%%%%%%%%%%%%%%%%%%%%%%%%%%%%%%

\usepackage{titlesec}

% \newcommand{\sectionbreak}[0]{\clearpage} % Section starts on a new page
% \renewcommand{\thesubsection}[0]{\thesection.\alph{subsection}}
% Subsections are letters

%\allowdisplaybreaks

% Removes section numbers
%\makeatletter
%\renewcommand{\@seccntformat}[1]{}
%\makeatother

%\setcounter{tocdepth}{0} % Set's table of Content depth, 0 = Chapters only
% To add Table of Conents Page
%\tableofcontents

%\appendix % Sets numbers to A.n after this point
%\chapter{Appendix}

%%%%%%%%%%%%%%%%%%%%%%%%%%%%%%%%%%%%%%%%%%%%%%%%%%%%%%%%%%%

\usepackage{amsfonts}
\usepackage{amsmath}

% If in \begin{align} mode, use \intertext{A comment} to add side text

\usepackage{amssymb}
\usepackage{mathtools}
\usepackage{cancel}
\usepackage{physics}

\usepackage{url}
\usepackage{hyperref}
\usepackage{cleveref} % cleveref has to go after hyperref
\usepackage{rotating}
\usepackage{fullpage}
%\usepackage{fixltx2e} - Now in Kernel?
% Normal text sub and super scripts

\usepackage[chapter, newfloat]{minted} % Copy and paste code in
% \begin{minted}[mathescape, linenos]{python}
% \end{minted}
% \inputminted[mathescape, linenos]{python}{FILENAME.py}

\usepackage{caption}

\newenvironment{code}{\captionsetup{type=listing}}{}
\SetupFloatingEnvironment{listing}{name=Code}

% Eg.
%\section{Program Output}\label{cod:outputNoZD}
%\begin{code}
%	\inputminted[breaklines]{text}{noZDOutput.txt}
%	\captionof{listing}{Console Output of My Own Tournament}
%\end{code}

\usepackage{graphicx} % Photo, Image
%\includegraphics[width=\textwidth]{FILE NAME}

\usepackage{placeins}
% Required for float barrier

% Eg.
%\FloatBarrier
%\begin{figure}[h]
%    \centering
%    \includegraphics[scale=0.45]{newBot.png}
%    \caption{Sample Bot Comment for Errors}
%    \label{fig:BotComment}
%\end{figure}
%\FloatBarrier

\usepackage{pdfpages} % Pdf
%\includepdf[page={page number}]{filename}
%\includepdf[page=-]{filename}

%%%%%%%%%%%%%%%%%%%%%%%%%%%%%%%%%%%%%%%%%%%%%%%%%%%%%%%%%%%

% Referencing
%\usepackage{biblatex}
% Add inbetween usepackage and {..} for frames (slides)
%[style=verbose,backend=biber, autocite=superscript]
%\addbibresource{ref.bib}
%\cite[pg n]{ Key }
% pg optional

% In ref.bib file
% Can use \begin{filecontents}{ref.bib} ... \end{filecontents}
% @Type{ Key, field = {value}, ... }

% Eg.
%@misc{AxelrodLibrary,
%  author = {Axelrod-Python},
%  title = {Axelrod},
%  year = {2020},
%  publisher = {GitHub},
%  journal = {GitHub repository},
%  howpublished = {\url{https://github.com/Axelrod-Python/Axelrod}},
%  commit = {c1c6a561602af2a42e2603581d04fb995ae87ac6}
%}

% Eg.
%@misc{mathWorldCramers,
%author = {Weisstein, Eric W.},
%journal = {MathWorld--A Wolfram Web Resource},
%publisher = {Wolfram Research, Inc.},
%title = {{Cramer's Rule}},
%howpublished = {\url{https://mathworld.wolfram.com/CramersRule.html}},
%note = {Accessed: 2021-01-11},
%}

%\bibliographystyle{unsrt}
%\bibliography{references}

\usepackage{fnpct} % Footcite size
%\setbeamerfont{footnote}{size=\scriptsize}

%%%%%%%%%%%%%%%%%%%%%%%%%%%%%%%%%%%%%%%%%%%%%%%%%%%%%%%%%%%

% Frames
%\frame{ ... }
%\frametitle{title}
%\onslide<1->{ ... }
%\footcite{ ... }

%%%%%%%%%%%%%%%%%%%%%%%%%%%%%%%%%%%%%%%%%%%%%%%%%%%%%%%%%%%

% .X + .Y = 1
% Works for n pages

%\begin{minipage}{.X\textwidth}
% ...
%\end{minipage}%
%\begin{minipage}{.Y\textwidth}
% ...
%\end{minipage}%

%%%%%%%%%%%%%%%%%%%%%%%%%%%%%%%%%%%%%%%%%%%%%%%%%%%%%%%%%%%

\usepackage{tikz} % Graphs
\usetikzlibrary{shapes.arrows,positioning}
%\begin{tikzpicture}
%    \node (n1) at (0,0) [circle, draw, label = 1]{a};
%    \node (n2) at (2,0) [circle, draw, label = 2]{b};
%    \draw[red] (n1) -- (n2);
%\end{tikzpicture}

%%%%%%%%%%%%%%%%%%%%%%%%%%%%%%%%%%%%%%%%%%%%%%%%%%%%%%%%%%%

\usepackage[english]{babel}
\usepackage[utf8]{inputenc}
\usepackage{fancyhdr}

\usepackage[
	headheight=120pt,% gap between header and text, gap between line and text
]{geometry}

% Add a gap, vertical space, break, new line
% \vspace{5mm}

\fancypagestyle{headerPageStyle}
{
   \fancyhf{}
   \cfoot{\thepage} % Page No. center
}

%%%%%%%%%%%%%%%%%%%%%%%%%%%%%%%%%%%%%%%%%%%%%%%%%%%%%%%%%%%

\usepackage{csquotes} % Quotes -Has to go after inputenc
%\blockquote{I ... end.}

%%%%%%%%%%%%%%%%%%%%%%%%%%%%%%%%%%%%%%%%%%%%%%%%%%%%%%%%%%%

\newcommand{\cid}[0]{01493016}
\newcommand{\lcid}[0]{CID: \cid}
\newcommand{\name}[0]{Ben David Pahnke}
\newcommand{\lname}[0]{Name: \name}
\newcommand{\degreeCode}[0]{GG41}
\newcommand{\ldegreeCode}[0]{Degree Code: \degreeCode}
\newcommand{\lcourseCode}[1]{Course Code: #1}

\newcommand{\nameHeader}[0]{\pagestyle{headerPageStyle}
\rhead{\name\\ \cid} \lhead{\leftmark}}
\newcommand{\degreeHeader}[0]{\nameHeader
\lhead{\ldegreeCode}}
\newcommand{\courseHeader}[1]{\nameHeader
\lhead{\lcourseCode{#1} \\ \ldegreeCode}}

%%%%%%%%%%%%%%%%%%%%%%%%%%%%%%%%%%%%%%%%%%%%%%%%%%%%%%%%%%%

\usepackage{amsthm}

% Use \begin{theorem} ... \end{theorem}
%\theoremstyle{definition}               % Make theorems & co not italics
\newtheorem{theorem}{Theorem}[chapter]  % Uncomment for theorem based off chapter no
                                         % Can use [section] and so on
\newtheorem{lemma}[theorem]{Lemma}       % Lemma numbers use theorem numbers
\newtheorem{corollary}[theorem]{Corollary}
\newtheorem{proposition}[theorem]{Proposition}
\newtheorem*{remark}{Remark}             % Remarks have no numbers
\newtheorem*{remarks}{Remarks}

% \label{labName}
% \ref{labName}                          % Give \cref{labName} a try

\usepackage{etoolbox}                    % http://ctan.org/pkg/etoolbox
% Changes definitions headings
\patchcmd{\thmhead}{(#3)}{: [\textbf{#3}]}{}{}
\patchcmd{\thmpuncuation}{.}{}{}{}
\newtheorem{definition}[theorem]{Definition}

\theoremstyle{definition}
\newtheorem*{example}{Example}

\usepackage{chngcntr}
%\counterwithin{equation}{theorem}        % Equation no's sub theorem no

%%%%%%%%%%%%%%%%%%%%%%%%%%%%%%%%%%%%%%%%%%%%%%%%%%%%%%%%%%%

\newcommand{\tLine}[0]{  \\ \hline } % New Table line

%%%%%%%%%%%%%%%%%%%%%%%%%%%%%%%%%%%%%%%%%%%%%%%%%%%%%%%%%%%

\newcommand{\topCorn}[1]{\ulcorner #1 \urcorner}
\newcommand{\sPPair}[2]{\langle \langle #1, #2 \rangle \rangle}
\newcommand{\sPair}[2]{\langle #1, #2 \rangle}
\newcommand{\defEq}[0]{\stackrel{\text{def}}{=}} % = with def above it

%%%%%%%%%%%%%%%%%%%%%%%%%%%%%%%%%%%%%%%%%%%%%%%%%%%%%%%%%%%

\newcommand{\intAB}[2]{\int\limits_{#1}^{#2}}     % Integration
\newcommand{\sumAB}[2]{\sum\limits_{#1}^{#2}}
\newcommand{\prodAB}[2]{\prod\limits_{#1}^{#2}}
\newcommand{\cupAB}[2]{\bigcup\limits_{#1}^{#2}}

%%%%%%%%%%%%%%%%%%%%%%%%%%%%%%%%%%%%%%%%%%%%%%%%%%%%%%%%%%%

\newcommand{\spTxt}[1]{\ \text{#1} \ }
\newcommand{\spImp}[0]{\ \implies \ }

%%%%%%%%%%%%%%%%%%%%%%%%%%%%%%%%%%%%%%%%%%%%%%%%%%%%%%%%%%%

\newcommand{\half}[0]{\frac{1}{2}}

%%%%%%%%%%%%%%%%%%%%%%%%%%%%%%%%%%%%%%%%%%%%%%%%%%%%%%%%%%%

%\abs{x} % |x|
\newcommand{\rang}[2]{ [#1  ... \ #2 ]\,} %Inclusive Range
%\newcommand{\eval}[1]{\begin{array}[t]{@{}c@{\,}|@{\,}}%
%\raisebox{0pt}[0.85\height][1.33\depth]{$ \displaystyle#1 $}\end{array}}
% Evaluation for differentiation -In a package now, same thing though

%%%%%%%%%%%%%%%%%%%%%%%%%%%%%%%%%%%%%%%%%%%%%%%%%%%%%%%%%%%

% Uppercase Latin letters
\newcommand{\tL}[0]{\text{L}}
\newcommand{\tP}[0]{\text{P}}
\newcommand{\tE}[0]{\text{E}}
\newcommand{\tQ}[0]{\text{Q}}
\newcommand{\tR}[0]{\text{R}}
\newcommand{\tS}[0]{\text{S}}
\newcommand{\tT}[0]{\text{T}}

% Lowercase Latin letters
\newcommand{\tp}[0]{\text{p}}

% "Phrases"
\newcommand{\st}[0]{\text{s.t}}
\newcommand{\sst}[0]{\ \st \ }

% Misc
\newcommand{\hpi}[0]{\hat{\pi}}

%%%%%%%%%%%%%%%%%%%%%%%%%%%%%%%%%%%%%%%%%%%%%%%%%%%%%%%%%%%

\newcommand{\mR}[0]{\mathbb{R}} % real numbers
\newcommand{\mZ}[0]{\mathbb{Z}} % integers
\newcommand{\mN}[0]{\mathbb{N}} % natural numbers
\newcommand{\mQ}[0]{\mathbb{Q}} % rational numbers
\newcommand{\mC}[0]{\mathbb{C}} % complex numbers

%%%%%%%%%%%%%%%%%%%%%%%%%%%%%%%%%%%%%%%%%%%%%%%%%%%%%%%%%%%

\newcommand{\betaFunc}[2]{ \frac{ \Gamma (#1 + #2)}{\Gamma(#1)\Gamma(#2)}} % Beta Function

%\underbrace{Above the bracket}_\text{Under the bracket}

% Function def
% f(x) = \begin{cases}
%X,& \text{Info}\\
%Y,& \text{otherwise}\\
%\end{cases}

% Matrix def
% \begin{pmatrix}
% 1 & 2 & 3 \\
% 4 & 5 & 6 \\
% \end{pmatrix}

% Isomorphism
% \cong

% cross x
%\times
% dot
%\cdot
% divide, div (caused by physics package)
%\divisionsymbol

% Tends to, limit
%\to

% Line, squiggle, wavy bar
%\tilde{z}

% ^ bar, line
%\hat{z}

% bold
% \textbf{z}
