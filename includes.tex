% \documentclass[12pt,twoside]{article}
% \usepackage{enumitem}% http://ctan.org/pkg/enumitem
\newcommand{\nItem}[0]{\newpage \item}

%%%%%%%%%%%%%%%%%%%%%%%%%%%%%%%%%%%%%%%%%%%%%%%%%%%%%%%%%%%

\usepackage{titlesec}

\titleformat{\section}[wrap]
{\normalfont\bfseries}
{\thesection.}{0.5em}{}
\titlespacing{\section}{12pc}{1.5ex plus .1ex minus .2ex}{1pc}

\newcommand{\sectionbreak}[0]{\clearpage} % Section starts on a new page
\renewcommand{\thesubsection}[0]{\thesection.\alph{subsection}} % Subsections are letters

%%%%%%%%%%%%%%%%%%%%%%%%%%%%%%%%%%%%%%%%%%%%%%%%%%%%%%%%%%%

\usepackage{amsfonts}
\usepackage{amsmath}
\usepackage{amssymb}
\usepackage{mathtools}
\usepackage{cancel}
\usepackage{tikz} % Graphs

\usepackage{hyperref}
\usepackage{rotating}
\usepackage{fullpage}

%%%%%%%%%%%%%%%%%%%%%%%%%%%%%%%%%%%%%%%%%%%%%%%%%%%%%%%%%%%

\usepackage[english]{babel}
\usepackage[utf8]{inputenc}
\usepackage{fancyhdr}

\fancypagestyle{firstPageStyle}
{
   \fancyhf{}
   \nameHeader

} % Used to set style of first page

%%%%%%%%%%%%%%%%%%%%%%%%%%%%%%%%%%%%%%%%%%%%%%%%%%%%%%%%%%%

\newcommand{\cid}[0]{01493016}
\newcommand{\lcid}[0]{CID: \cid}
\newcommand{\name}[0]{Ben David Pahnke}
\newcommand{\lname}[0]{Name: \name}
\newcommand{\degreeCode}[0]{GG41}
\newcommand{\ldegreeCode}[0]{Degree Code: \degreeCode}
\newcommand{\lcourseCode}[1]{Course Code: #1}

\newcommand{\nameHeader}[0]{\pagestyle{fancy}
\rhead{\name\\ \cid}}
\newcommand{\courseHeader}[1]{\nameHeader
\lhead{\lcourseCode{#1}}}
\newcommand{\degreeHeader}[1]{\nameHeader
\lhead{\lcourseCode{#1} \\
\ldegreeCode}}

%%%%%%%%%%%%%%%%%%%%%%%%%%%%%%%%%%%%%%%%%%%%%%%%%%%%%%%%%%%

\newcommand{\tLine}[0]{  \\ \hline } % New Table line

%%%%%%%%%%%%%%%%%%%%%%%%%%%%%%%%%%%%%%%%%%%%%%%%%%%%%%%%%%%

\newcommand{\topCorn}[1]{\ulcorner #1 \urcorner}
\newcommand{\sPPair}[2]{\langle \langle #1, #2 \rangle \rangle}
\newcommand{\sPair}[2]{\langle #1, #2 \rangle}
\newcommand{\defEq}[0]{\stackrel{\text{def}}{=}} % = with def above it

%%%%%%%%%%%%%%%%%%%%%%%%%%%%%%%%%%%%%%%%%%%%%%%%%%%%%%%%%%%

\newcommand{\intAB}[2]{\int\limits^{#1}_{#2}}
\newcommand{\sumAB}[2]{\sum\limits_{#1}^{#2}}
\newcommand{\cupAB}[2]{\bigcup\limits_{#1}^{#2}}

%%%%%%%%%%%%%%%%%%%%%%%%%%%%%%%%%%%%%%%%%%%%%%%%%%%%%%%%%%%

\newcommand{\spTxt}[1]{\ \text{#1} \ }
\newcommand{\spImp}[0]{\ \implies \ }

%%%%%%%%%%%%%%%%%%%%%%%%%%%%%%%%%%%%%%%%%%%%%%%%%%%%%%%%%%%

\newcommand{\half}[0]{\frac{1}{2}}

%%%%%%%%%%%%%%%%%%%%%%%%%%%%%%%%%%%%%%%%%%%%%%%%%%%%%%%%%%%

\newcommand{\rang}[2]{ [#1  ... \ #2 ]\,} %Inclusive Range

%%%%%%%%%%%%%%%%%%%%%%%%%%%%%%%%%%%%%%%%%%%%%%%%%%%%%%%%%%%

\newcommand{\mR}[0]{\mathbb{R}} % real numbers
\newcommand{\mZ}[0]{\mathbb{Z}} % integers
\newcommand{\mN}[0]{\mathbb{N}} % natural numbers
\newcommand{\mQ}[0]{\mathbb{Q}} % rational numbers
\newcommand{\mC}[0]{\mathbb{C}} % complex numbers

%%%%%%%%%%%%%%%%%%%%%%%%%%%%%%%%%%%%%%%%%%%%%%%%%%%%%%%%%%%

\newcommand{\betaFunc}[2]{ \frac{ \Gamma (#1 + #2)}{\Gamma(#1)\Gamma(#2)}} % Beta Function

%
% \begin{document}
%
%
% \end{document}
% Used to import includes

%\ll to start vim compiling
%\le for error messages

\usepackage{titlesec}

%\newcommand{\sectionbreak}[0]{\clearpage} % Section starts on a new page
% \renewcommand{\thesubsection}[0]{\thesection.\alph{subsection}}
% Subsections are letters

%\allowdisplaybreaks

% Removes section numbers
%\makeatletter
%\renewcommand{\@seccntformat}[1]{}
%\makeatother

%\setcounter{tocdepth}{0} % Set's table of Content depth, 0 = Chapters only
% To add Table of Conents Page
%\tableofcontents

%\appendix % Sets numbers to A.n after this point
%\chapter{Appendix}

%%%%%%%%%%%%%%%%%%%%%%%%%%%%%%%%%%%%%%%%%%%%%%%%%%%%%%%%%%%

\usepackage{amsfonts}
\usepackage{amsmath}

\usepackage{amssymb}
\usepackage{mathtools}
\usepackage{cancel}
\usepackage{physics}

\usepackage{url}
\usepackage{rotating}
\usepackage{fullpage}

%\usepackage{fixltx2e} - Now in Kernel?
% Normal text sub and super scripts

%\usepackage[chapter, newfloat]{minted} % Copy and paste code in
% \begin{minted}[mathescape, linenos]{python}
% \end{minted}
% \inputminted[mathescape, linenos]{python}{FILENAME.py}

\usepackage[justification=centering]{caption}

%\newenvironment{code}{\captionsetup{type=listing}}{}
%\SetupFloatingEnvironment{listing}{name=Code}

% Eg.
%\section{Program Output}\label{cod:outputNoZD}
%\begin{code}
%	\inputminted[breaklines]{text}{noZDOutput.txt}
%	\captionof{listing}{Console Output of My Own Tournament}
%\end{code}

\usepackage{graphicx} % Photo, Image
\usepackage{lipsum}
%\includegraphics[width=\textwidth]{FILE NAME}

\usepackage{placeins}
% Required for float barrier

% Eg.
%\FloatBarrier
%\begin{figure}[h]
%    \centering
%    \includegraphics[scale=0.45]{newBot.png}
%    \caption{Sample Bot Comment for Errors}
%    \label{fig:BotComment}
%\end{figure}
%\FloatBarrier

\usepackage{pdfpages} % Pdf
%\includepdf[page={page number}]{filename}
%\includepdf[page=-]{filename}


%%%%%%%%%%%%%%%%%%%%%%%%%%%%%%%%%%%%%%%%%%%%%%%%%%%%%%%%%%%

% Referencing, Bibliography
%\usepackage{biblatex}
\usepackage[numbers]{natbib}
% Add inbetween usepackage and {..} for frames (slides)
%[style=verbose,backend=biber, autocite=superscript]
%\addbibresource{ref.bib}
%\cite[pg n]{ Key }
% pg optional

% In ref.bib file
% Can use \begin{filecontents}{ref.bib} ... \end{filecontents}
% @Type{ Key, field = {value}, ... }

% Eg.
%@misc{AxelrodLibrary,
%  author = {Axelrod-Python},
%  title = {Axelrod},
%  year = {2020},
%  publisher = {GitHub},
%  journal = {GitHub repository},
%  howpublished = {\url{https://github.com/Axelrod-Python/Axelrod}},
%  commit = {c1c6a561602af2a42e2603581d04fb995ae87ac6}
%}

% Eg.
%@misc{mathWorldCramers,
%author = {Weisstein, Eric W.},
%journal = {MathWorld--A Wolfram Web Resource},
%publisher = {Wolfram Research, Inc.},
%title = {{Cramer's Rule}},
%howpublished = {\url{https://mathworld.wolfram.com/CramersRule.html}},
%note = {Accessed: 2021-01-11},
%}

%\bibliographystyle{unsrt}
%\bibliography{references}

\usepackage{fnpct} % Footcite size
%\setbeamerfont{footnote}{size=\scriptsize}

%%%%%%%%%%%%%%%%%%%%%%%%%%%%%%%%%%%%%%%%%%%%%%%%%%%%%%%%%%%

% Frames
%\frame{ ... }
%\frametitle{title}
%\onslide<1->{ ... }
%\footcite{ ... }

% Footnote - non-beamer
%\footnote{Text for footnote}

%%%%%%%%%%%%%%%%%%%%%%%%%%%%%%%%%%%%%%%%%%%%%%%%%%%%%%%%%%%

% .X + .Y = 1
% Works for n pages

%\begin{minipage}{.X\textwidth}
% ...
%\end{minipage}%
%\begin{minipage}{.Y\textwidth}
% ...
%\end{minipage}%

%%%%%%%%%%%%%%%%%%%%%%%%%%%%%%%%%%%%%%%%%%%%%%%%%%%%%%%%%%%

\usepackage{tikz} % Graphs
\usetikzlibrary{shapes.arrows,positioning}
%\begin{tikzpicture}
%    \node (n1) at (0,0) [circle, draw, label = 1]{a};
%    \node (n2) at (2,0) [circle, draw, label = 2]{b};
%    \draw[red] (n1) -- (n2);
%\end{tikzpicture}

%%%%%%%%%%%%%%%%%%%%%%%%%%%%%%%%%%%%%%%%%%%%%%%%%%%%%%%%%%%

%\usepackage[english]{babel}
%\usepackage[utf8]{inputenc}
\usepackage{fancyhdr}

%\usepackage[
%	headheight=120pt,% gap between header and text, gap between line and text
%]{geometry}

% Add a gap, vertical space, break, new line
% \vspace{5mm}

%\fancypagestyle{headerPageStyle}
%{
%   \fancyhf{}
%   \cfoot{\thepage} % Page No. center
%}

%%%%%%%%%%%%%%%%%%%%%%%%%%%%%%%%%%%%%%%%%%%%%%%%%%%%%%%%%%%


%%%%%%%%%%%%%%%%%%%%%%%%%%%%%%%%%%%%%%%%%%%%%%%%%%%%%%%%%%%

\newcommand{\cid}[0]{01493016}
\newcommand{\lcid}[0]{CID: \cid}
\newcommand{\name}[0]{Ben David Pahnke}
\newcommand{\lname}[0]{Name: \name}
\newcommand{\degreeCode}[0]{GG41}
\newcommand{\ldegreeCode}[0]{Degree Code: \degreeCode}
\newcommand{\lcourseCode}[1]{Course Code: #1}

\newcommand{\nameHeader}[0]{\pagestyle{headerPageStyle}
\rhead{\name\\ \cid} \lhead{\leftmark}}
\newcommand{\degreeHeader}[0]{\nameHeader
\lhead{\ldegreeCode}}
\newcommand{\courseHeader}[1]{\nameHeader
\lhead{\lcourseCode{#1} \\ \ldegreeCode}}
\newcommand{\courseNoDegHeader}[1]{\nameHeader
\lhead{#1}}

\newcommand{\anonBorder}[2]{\pagestyle{headerPageStyle}
\lhead{#1}\rhead{#2}}

%%%%%%%%%%%%%%%%%%%%%%%%%%%%%%%%%%%%%%%%%%%%%%%%%%%%%%%%%%%

\usepackage{amsthm}

% Use \begin{theorem} ... \end{theorem}
%\theoremstyle{definition}               % Make theorems & co not italics
%\newtheorem{theorem}{Theorem}[chapter]  % Uncomment for theorem based off chapter no
                                         % Can use [section] and so on
%\newtheorem{lemma}[theorem]{Lemma}       % Lemma numbers use theorem numbers
%\newtheorem{corollary}[theorem]{Corollary}
%\newtheorem{proposition}[theorem]{Proposition}
%\newtheorem*{remark}{Remark}             % Remarks have no numbers
%\newtheorem*{remarks}{Remarks}

% \label{labName}
% \ref{labName}                          % Give \cref{labName} a try

\usepackage{etoolbox}                    % http://ctan.org/pkg/etoolbox
% Changes definitions headings
\patchcmd{\thmhead}{(#3)}{: [\textbf{#3}]}{}{}
\patchcmd{\thmpuncuation}{.}{}{}{}
%\newtheorem{definition}[theorem]{Definition}

\theoremstyle{definition}
%\newtheorem*{example}{Example}

\usepackage{chngcntr}
%\counterwithin{equation}{theorem}        % Equation no's sub theorem no

%%%%%%%%%%%%%%%%%%%%%%%%%%%%%%%%%%%%%%%%%%%%%%%%%%%%%%%%%%%

% New Table line
\newcommand{\tLine}[0]{  \\ \hline }
% Table
% \begin{center}
% \begin{tabular}{ |p|p|p| }
%  \hline
%  cell1 & cell2 & cell3 \\
%  cell4 & cell5 & cell6 \\
%  cell7 & cell8 & cell9 \\
%  \hline
% \end{tabular}
% \end{center}

% Diagonal Box, Split Box
\usepackage{diagbox}
% \backslashbox{Room}{Date}

% Cell Newline Wrap Text in Cell
\usepackage{makecell}
% \makecell{First \\ Second, all in one cell}

% Fixed Width Column
\usepackage{array}
% \begin{tabular}{ |m||m{5cm}|m|m| }

\usepackage{longtable}

%%%%%%%%%%%%%%%%%%%%%%%%%%%%%%%%%%%%%%%%%%%%%%%%%%%%%%%%%%%

\newcommand{\topCorn}[1]{\ulcorner #1 \urcorner}
\newcommand{\botCorn}[1]{\llcorner #1 \lrcorner}
\newcommand{\sPPair}[2]{\langle \langle #1, #2 \rangle \rangle}
\newcommand{\sPair}[2]{\langle #1, #2 \rangle}
\newcommand{\defEq}[0]{\stackrel{\text{def}}{=}} % = with def above it

%%%%%%%%%%%%%%%%%%%%%%%%%%%%%%%%%%%%%%%%%%%%%%%%%%%%%%%%%%%

% Better Limits ab
% \sumAB{i=1}{n}
\newcommand{\intAB}[2]{\int\limits_{#1}^{#2}}     % Integration
\newcommand{\sumAB}[2]{\sum\limits_{#1}^{#2}}
\newcommand{\prodAB}[2]{\prod\limits_{#1}^{#2}}
\newcommand{\cupAB}[2]{\bigcup\limits_{#1}^{#2}} % Union union
\newcommand{\capAB}[2]{\bigcap\limits_{#1}^{#2}} % Intersection intersection
\newcommand{\oPlusAB}[2]{\bigoplus\limits_{#1}^{#2}}
\newcommand{\oTimesAB}[2]{\bigotimes\limits_{#1}^{#2}}

% Tensor Product tensor product o cross ocross
% \otimes
% \bigotimes

% Direct Sum direct sum plus in circle Circle Plus
% \oplus
% \bigoplus

%%%%%%%%%%%%%%%%%%%%%%%%%%%%%%%%%%%%%%%%%%%%%%%%%%%%%%%%%%%

% Spaces
\newcommand{\spTxt}[1]{\ \text{#1} \ }
\newcommand{\spImp}[0]{\ \implies \ }
% Comma Dots ", ..., "
\newcommand{\comdots}[0]{,\, \dots,\,}

%%%%%%%%%%%%%%%%%%%%%%%%%%%%%%%%%%%%%%%%%%%%%%%%%%%%%%%%%%%

% Fractions
\newcommand{\half}[0]{\frac{1}{2}}
\newcommand{\quarter}[0]{\frac{1}{4}}

%%%%%%%%%%%%%%%%%%%%%%%%%%%%%%%%%%%%%%%%%%%%%%%%%%%%%%%%%%%

% Ranges
%\abs{x} % |x|
\newcommand{\rang}[2]{ [#1  ... \ #2 ]\,} %Inclusive Range
%\newcommand{\eval}[1]{\begin{array}[t]{@{}c@{\,}|@{\,}}%
%\raisebox{0pt}[0.85\height][1.33\depth]{$ \displaystyle#1 $}\end{array}}
% Evaluation for differentiation -In a package now, same thing though

%%%%%%%%%%%%%%%%%%%%%%%%%%%%%%%%%%%%%%%%%%%%%%%%%%%%%%%%%%%

% Uppercase Latin letters
\newcommand{\tL}[0]{\text{L}}
\newcommand{\tP}[0]{\text{P}}
\newcommand{\tE}[0]{\text{E}}
\newcommand{\tQ}[0]{\text{Q}}
\newcommand{\tR}[0]{\text{R}}
\newcommand{\tS}[0]{\text{S}}
\newcommand{\tT}[0]{\text{T}}

% Lowercase Latin letters
\newcommand{\tp}[0]{\text{p}}

% Bold Letters Numbers Vector
\newcommand{\vZero}[0]{\mathbf{0}}
\newcommand{\vOne}[0]{\mathbf{1}}
\newcommand{\vk}[0]{\mathbf{k}}
\newcommand{\vM}[0]{\mathbf{M}}
\newcommand{\vN}[0]{\mathbf{N}}
\newcommand{\vn}[0]{\mathbf{n}}
\newcommand{\vS}[0]{\mathbf{S}}
\newcommand{\vQ}[0]{\mathbf{Q}}
\newcommand{\vx}[0]{\mathbf{x}}
\newcommand{\bu}[0]{\textbf{\textit{u}}}

% "Phrases"
% Such that such that
\newcommand{\st}[0]{\text{s.t}}
\newcommand{\sst}[0]{\ \st \ }
\newcommand{\spst}[0]{\ \st \ }

\newcommand{\tnot}[0]{\text{not }}
\DeclareMathOperator{\prob}{Prob}

% Misc
\newcommand{\hpi}[0]{\hat{\pi}}

% Gothic gothic letters
\newcommand{\ga}[0]{\mathfrak{a}}

% Means, Average
\newcommand{\oN}[0]{\overline N}
\newcommand{\oNi}[0]{\overline N_i}
\newcommand{\oNic}[0]{\overline N_{i,\, c}}
\newcommand{\oNik}[0]{\overline N_{i,\, k}}
\newcommand{\oNis}[0]{\overline N_{i,\, s}}
\newcommand{\oQ}[0]{\overline Q}
\newcommand{\oQi}[0]{\overline Q_i}
\newcommand{\oQic}[0]{\overline Q_{i,\, c}}
\newcommand{\oQik}[0]{\overline Q_{i,\, k}}
\newcommand{\oR}[0]{\overline R}
\newcommand{\oRi}[0]{\overline R_i}
\newcommand{\oRic}[0]{\overline R_{i,\, c}}
\newcommand{\oRik}[0]{\overline R_{i,\, k}}

% Bold Maths Letters
\newcommand{\bC}{\mathbf{C}}

% Infinity
% \infty

%%%%%%%%%%%%%%%%%%%%%%%%%%%%%%%%%%%%%%%%%%%%%%%%%%%%%%%%%%%

% Numbers
\newcommand{\mR}[0]{\mathbb{R}} % Real
\newcommand{\mZ}[0]{\mathbb{Z}} % Integers
\newcommand{\mN}[0]{\mathbb{N}} % Natural
\newcommand{\mQ}[0]{\mathbb{Q}} % Rational
\newcommand{\mC}[0]{\mathbb{C}} % Complex

% Rings, fields
\newcommand{\mZi}[0]{\mathbb{Z}[i]}
% Ring of integers O ring Integers
\newcommand{\rO}[0]{\mathcal{O}}
\newcommand{\OK}[0]{\rO_K}

%%%%%%%%%%%%%%%%%%%%%%%%%%%%%%%%%%%%%%%%%%%%%%%%%%%%%%%%%%%

% Beta Function
\newcommand{\betaFunc}[2]{ \frac{ \Gamma (#1 + #2)}{\Gamma(#1)\Gamma(#2)}}

%\underbrace{Above the bracket}_\text{Under the bracket}

% If in \begin{align} mode, use \intertext{A comment} to add side text

% Function def
% f(x) = \begin{cases}
%X,& \text{Info}\\
%Y,& \text{otherwise}\\
%\end{cases}

% Matrix def
% \begin{pmatrix}
% 1 & 2 & 3 \\
% 4 & 5 & 6 \\
% \end{pmatrix}

% Isomorphism iso
% \cong
\newcommand{\iso}[0]{\cong}

% Mod, mod, modulo, Congurent
% \equiv

% Cross x
% \times
% Dot
% \cdot

% Divide, div (caused by physics package)
% \divisionsymbol

% Doesn't divide Line with bar
% \nmid

% plus minus, plus-minus, Plus Minus
% \pm

% Unit ring
% \cross
% Eg. A^{\cross}

% Tends to, Limit
% \to

% Line, Squiggle, Wavy Bar
% \tilde{z}

% ^ Bar, Line
% \hat{z}

% Bold
% \textbf{z}

% Subset sub set
% \subset
% \subseteq
% \subsetneq
% right to left
% \supset
% \supseteq
% \supsetneq

% Backslash back slash
\newcommand{\bsl}[0]{\char `\\}

% Arrow rightarrow leftarrow
% \leftarrow
% \rightarrow
% \mapsto
% \hooksrightarrow
% \longleftarrow

% Entails
% \models
% \not\models
\newcommand{\entail}[0]{\models}
\newcommand{\nentail}[0]{\not\models}
\newcommand{\entails}[0]{\models}
\newcommand{\nentails}[0]{\not\models}

% And \land or can use \wedge
% Or \lor or can use \vee

% Set comprehensions
\newcommand{\scomp}[2]{\left\{#1\ \mid \ #2\right\}}
% is an ideal of Ideal
\newcommand{\ideal}[0]{\triangleleft}
% is a prime ideal of Prime
\newcommand{\pideal}[0]{\triangleleft_{\mathfrak{p}}}
% is a maximal ideal of Maximal
\newcommand{\mideal}[0]{\triangleleft_{\mathfrak{m}}}

% Weird L l ell
% \ell

% Code Font
\newcommand{\code}[1]{\texttt{#1}}

%%%%%%%%%%%%%%%%%%%%%%%%%%%%%%%%%%%%%%%%%%%%%%%%%%%%%%%%%%%

\newcommand{\CL}[0]{\textit{CL}}
\newcommand{\BKT}[0]{\textit{BKT}}
\newcommand{\NLI}[0]{\textit{NLI}}
\newcommand{\RNLI}[0]{\textit{RNLI}}



%%%%%%%%%%%%%%%%%%%%%%%%%%%%%%%%%%%%%%%%%%%%%%%%%%%%%%%%%%%
\usepackage{multicol}

\usepackage{hyperref}
\usepackage[capitalise]{cleveref} % cleveref has to go after hyperref

\usepackage{enumitem}% http://ctan.org/pkg/enumitem
\usepackage{ulem}% Used to Underline the Step
\newcommand{\nItem}[0]{\newpage \item}
\newcommand{\nitem}[0]{\nItem}
\newcommand{\letEnum}[0]{\begin{enumerate}[label = \alph*)]}
\newcommand{\numEnum}[0]{\begin{enumerate}[label = \arabic*.]}
\newcommand{\romEnum}[0]{\begin{enumerate}[label = \roman*)]}
\newcommand{\eEnum}[0]{\end{enumerate}}

\newlist{steps}{enumerate}{3}
\setlist[steps,1]{label={\uline{Step \arabic*:}}, ref={\arabic*}}
\setlist[steps,2]{label={\uline{Step \thestepsi.\arabic*:}}, ref={\thestepsi.\arabic*}}
\setlist[steps,3]{label={\uline{Step \thestepsii.\arabic*:}}, ref={\thestepsii.\arabic*}}
\crefname{stepsi}{Step}{Steps}
\Crefname{stepsi}{Step}{Steps}
\crefname{stepsii}{Step}{Steps}
\Crefname{stepsii}{Step}{Steps}
\crefname{stepsiii}{Step}{Steps}
\Crefname{stepsiii}{Step}{Steps}
\newcommand{\startSteps}[0]{\begin{steps}}
\newcommand{\step}[0]{\item \mbox{}\\}
\newcommand{\stopSteps}[0]{\end{steps}}

% For Bullet points use:
%\begin{itemize}
%		\item
%\end{itemize}

%%%%%%%%%%%%%%%%%%%%%%%%%%%%%%%%%%%%%%%%%%%%%%%%%%%%%%%%%%%

\usepackage{pdflscape}
\usepackage{afterpage}

\usepackage[section]{placeins}


%%%%%%%%%%%%%%%%%%%%%%%%%%%%%%%%%%%%%%%%%%%%%%%%%%%%%%%%%%%

% Beamer, Presentation, Slides

%\begin{frame}
%	\frametitle{Two Figures aside}
%	\begin{columns}[b]
%		\column{0.0125\textwidth}
%		\column{0.4875\textwidth}
%			\centering
%			\begin{figure}
%				\includegraphics[draft,width=\textwidth]{fooA.pdf} \
%				% remove the 'draft' keyword, when replacing with final figure!
%				\caption{Caption of Figure A}
%			\end{figure}
%		\column{0.4875\textwidth}
%			\centering
%			\begin{figure}
%				\includegraphics[draft,width=\textwidth]{fooB.pdf} \
%				% remove the 'draft' keyword, when replacing with final figure!
%				\caption{Caption of Figure B}
%			\end{figure}
%		\column{0.0125\textwidth}
%	\end{columns}
%\end{frame}
%
%
%\begin{frame}
%	\frametitle{Text and Figure aside}
%	\begin{columns}[]
%		\column{0.0125\textwidth}
%		\column{0.4875\textwidth}
%			\centering
%			\begin{figure}
%				\includegraphics[draft,width=\textwidth]{fooC.pdf} \
%				% remove the 'draft' keyword, when replacing with final figure!
%				\caption{Caption of Figure C}
%			\end{figure}
%		\column{0.4875\textwidth}
%			Some text and a bullet point list
%			\begin{itemize}
%				\item ItemA
%				\item ItemB
%				\item ItemC
%				\item ItemD
%			\end{itemize}
%		\column{0.0125\textwidth}
%	\end{columns}
%\end{frame}
%
%
%\begin{frame}
%	\frametitle{One Figure}
%	\bigskip
%	\begin{figure}
%		\includegraphics[draft,width=0.8\textwidth, height=0.5\textwidth]{fooD.pdf}
%		% remove the 'draft' keyword, when replacing with final figure!
%		\caption{Caption of Figure D}
%	\end{figure}
%\end{frame}
%
%
%\begin{frame}
%	\frametitle{Table and Block}
%	\begin{table}[h]
%		\centering
%		\begin{tabular}{p{0.2\textwidth} p{0.7\textwidth}}
%			\toprule
%			\multicolumn{2}{p{0.9\textwidth}}{TableTitle} \\
%			\midrule
%			CapA       & lots of examplesA \\
%
%			CapB       & lots of examplesB \\
%
%			CapC       & lots of examplesC \\
%			\bottomrule
%		\end{tabular}
%	\end{table}
%
%	\begin{block}{BlockTitle}
%		something to emphasise
%	\end{block}
%\end{frame}
